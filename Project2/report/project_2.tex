\documentclass{article}
\usepackage[utf8]{inputenc}

\usepackage{listings}
\usepackage{color}
\usepackage{graphicx} %package to manage images

%New colors defined below
\definecolor{codegreen}{rgb}{0,0.6,0}
\definecolor{codegray}{rgb}{0.5,0.5,0.5}
\definecolor{codepurple}{rgb}{0.58,0,0.82}
\definecolor{backcolour}{rgb}{0.95,0.95,0.92}

%Code listing style named "mystyle"
\lstdefinestyle{mystyle}{
  backgroundcolor=\color{backcolour},   commentstyle=\color{codegreen},
  keywordstyle=\color{magenta},
  numberstyle=\tiny\color{codegray},
  stringstyle=\color{codepurple},
  basicstyle=\footnotesize,
  breakatwhitespace=false,         
  breaklines=false,                 
  captionpos=b,                    
  keepspaces=true,                 
  numbers=left,                    
  numbersep=5pt,                  
  showspaces=false,                
  showstringspaces=false,
  showtabs=false,                  
  tabsize=2
}

%"mystyle" code listing set
\lstset{style=mystyle}

\begin{document}
\centerline{\sc \large EECE 7360 Project 2}
\vspace{.5pc}
\centerline{\sc Garrett Goode and Daniel Hullihen}
\centerline{\it Spring 2017}

\section{Introduction}
The subset sum problem (also referred to as the ``exact knapsack problem'')
is defined below.

\textit{Let A = $\{a_1, ..., a_n\}$ represent some set of integers. Given a sum s, find a subset $A' \subset A$ such that
  $$s = \sum_{i=1}^n a_i, for 1 \le i \le n.$$}
In other words, if we are given a list of numbers and some target sum, find the numbers in the list that
add up to the target sum. 

In this project, we explored the complexity landscape around the subset sum problem, including
subproblems to subset sum, as well as problems of which which subset sum itself is a subproblem.

\section{Natural NPC Subproblems of Subset Sum}

\section{Natural NPC Problems of which Subset Sum is a Subproblem}

\section{Summary}

\bibliography{references}
\bibliographystyle{plain}

\end{document}
