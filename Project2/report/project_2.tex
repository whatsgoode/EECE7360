\documentclass{article}
\usepackage[utf8]{inputenc}
\usepackage{amssymb}

\usepackage{listings}
\usepackage{color}
\usepackage{graphicx} %package to manage images

%New colors defined below
\definecolor{codegreen}{rgb}{0,0.6,0}
\definecolor{codegray}{rgb}{0.5,0.5,0.5}
\definecolor{codepurple}{rgb}{0.58,0,0.82}
\definecolor{backcolour}{rgb}{0.95,0.95,0.92}

%Code listing style named "mystyle"
\lstdefinestyle{mystyle}{
  backgroundcolor=\color{backcolour},   commentstyle=\color{codegreen},
  keywordstyle=\color{magenta},
  numberstyle=\tiny\color{codegray},
  stringstyle=\color{codepurple},
  basicstyle=\footnotesize,
  breakatwhitespace=false,         
  breaklines=false,                 
  captionpos=b,                    
  keepspaces=true,                 
  numbers=left,                    
  numbersep=5pt,                  
  showspaces=false,                
  showstringspaces=false,
  showtabs=false,                  
  tabsize=2
}

%"mystyle" code listing set
\lstset{style=mystyle}

\begin{document}
\centerline{\sc \large EECE 7360 Project 2}
\vspace{.5pc}
\centerline{\sc Garrett Goode and Daniel Hullihen}
\centerline{\it Spring 2017}

\section{Introduction}
The subset sum problem (also referred to as the ``exact knapsack problem'')
is defined below.

\textit{Let A = $\{a_1, ..., a_n\}$ represent some set of integers. Given a sum s, find a subset $A' \subset A$ such that
  $$s = \sum_{i=1}^n a_i, for 1 \le i \le n.$$}
In other words, if we are given a list of numbers and some target sum, find the numbers in the list that
add up to the target sum. 

In this project, we explored the complexity landscape around the subset sum
problem, including subproblems to subset sum, as well as problems of which
which subset sum is itself a subproblem.

\section{Natural Subproblems of Subset Sum}
The subset sum problem can be solved in pseudo-polynomial time, which means
there are subproblems to subset sum that can be solved in polynomial time,
but the ``hardest'' subproblem in subset sum is nevertheless NP-Complete. Due
to this property, it is possible to detect some subproblems to subset
sum and apply a known algorithm that solves the problem in polynomial time.
Some algorithms used for solving subset sum use a divide-and-conquer
approach and identify these subproblems. Subproblems
to subset sum, for example, can have all numbers that have some special
property, or the set of numbers itself has some property that can be
exploited.
This section explores some of the natural subproblems to subset sum. 

\subsection{Sets with a low density}
One metric for describing an instance of subset sum is
the density of the set. The following equation
describes the density of a set.
$$d = n / log_2 max(a_i)$$
where n is the size of the set in quation and $max(a_i)$ is the largest
element in the set.
Lagarias and Odlyzko found that, for sets that have a density less than 0.645,
their ``Algorithm SV'' can solve the instance in polynomial
time \cite{lagarias1985}. One of the steps in their algorithm involves
taking set and perform a transformation in order to apply an
algorithm developed by Lenstra, Lenstra, and Lovasz, which has a time
complexity of $O(n^{12}+n^9(log|f|^3))$ \cite{lenstra1982}. The transformed
problem is one where a short vector e must be found within an integer lattice
L = L(a,M)

\subsection{Sets where all the numbers are coprime to m}
This approach (as well as the prior approach) focus on sets that are finite
cyclic groups, defined as $\mathbb{Z}_m = {0, 1, ..., m-1}$ of order m.
Koiliaris and Xu argue that, given a set $S \subseteq U(\mathbb{Z})_m$, which
describes a set of integers that are coprime to some integer m, finding
the set of all subset sums can be donyge with a time complexity of
$O(min(\sqrt{n}m,m^{5/4})log m log n)$ \cite{koiliaris2016}.
Note that an integer x is
coprime to m if the greatest common denomintor of the two values is 1.
As Koiliaris explains, the method behind the speed-up with this approach
is the ability to partition the set into different subsets such that ``every
such subset is contained in an arithmetic progression of the form
x, 2x, ..., lx''. With this, one can calculate the subset sums very quickly
by scaling l accordingly. These sums can then be added together. This is
only doable if m is a prime number, or if all the numbers are relative prime
to m.

\subsection{Set is a subset of $Z_m$}
Koiliaris and Xu are able to take the previous case and take it a step further,
developing an algorithm that finds all of the subsets of a set S in with a
time complexity of $O(min(\sqrt{n}m,m^{5/4})log^2m)$  \cite{koiliaris2016}.
This is for sets $S \subseteq \mathbb{Z}_m$, which is slightly different from
the previous case in that the elements of the set no longer have to be
coprime to some integer m.
Their algorithm involves recursively computing a partial subset sum
as they work across subsets of the original set.

\subsection{Set size $m \geq l^{1/\alpha}$ and elements are distinct}
Chaimovich, Freiman and Galil found that, if they took the subset sum
problem and restricted it such that $max{a_i | a \epsilon A} \leq l \leq
m^\alpha)$ (where l is some bound and m is the number of variables in the set),
then the instance could be solved with a time complexity of $O(lm^2)$
\cite{chaimovich1989}. One unique
feature of their algorithm is that supports large sets (previous algorithms
could only handle moderately large sets). The main approach in their algorithm
is, assuming $A^*$ is defined as ${S_b | B \subseteq A}$, and $S_b$ is
defined as $\sum_{a_i\epsilon B} a_i$, characterize $A^*$ as ``a small
collection of
arithmetic progressions.'' 

\section{Natural Problems of which Subset Sum is a Subproblem}
Partition (transformation to Subset Sum)
Knapsack (transformation to Partition)
3-dimensional matching (transformation to Partition)
Multiple subset sum 

\section{Summary}

\bibliography{references}
\bibliographystyle{plain}

\end{document}
