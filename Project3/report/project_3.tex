\documentclass{article}
\usepackage[utf8]{inputenc}

\usepackage{listings}
\usepackage{algorithm}
\usepackage{color}
\usepackage{graphicx} %package to manage images

%New colors defined below
\definecolor{codegreen}{rgb}{0,0.6,0}
\definecolor{codegray}{rgb}{0.5,0.5,0.5}
\definecolor{codepurple}{rgb}{0.58,0,0.82}
\definecolor{backcolour}{rgb}{0.95,0.95,0.92}

%Code listing style named "mystyle"
\lstdefinestyle{mystyle}{
  backgroundcolor=\color{backcolour},   commentstyle=\color{codegreen},
  keywordstyle=\color{magenta},
  numberstyle=\tiny\color{codegray},
  stringstyle=\color{codepurple},
  basicstyle=\footnotesize,
  breakatwhitespace=false,         
  breaklines=false,                 
  captionpos=b,                    
  keepspaces=true,                 
  numbers=left,                    
  numbersep=5pt,                  
  showspaces=false,                
  showstringspaces=false,
  showtabs=false,                  
  tabsize=2
}

%"mystyle" code listing set
\lstset{style=mystyle}

\begin{document}
\centerline{\sc \large EECE 7360 Project 3}
\vspace{.5pc}
\centerline{\sc Garrett Goode and Daniel Hullihen}
\centerline{\it Spring 2017}

\section{Introduction}
The subset sum problem (also referred to as the ``exact knapsack problem'')
is defined below.

\textit{Let A = $\{a_1, ..., a_n\}$ represent some set of integers. Given a sum s, find a subset $A' \subset A$ such that
  $$s = \sum_{i=1}^n a_i, for 1 \le i \le n.$$}
In other words, if we are given a list of numbers and some target sum, we want
to find the numbers in the list that would add up to the target sum. Put as a decision
problem, the question would be ``Is there a subset A' of A where the sum of the
elements of A' is s?''

In this project, we developed an implementation of a greedy algorithm
for the subset sum problem, and ran it against a suite of instances to evaluate
the performance of this algorithm.

\section{Greedy Algorithm Implementation}
The implementation of the greedy algorithm that was used for this project is relatively simple.

\begin{algorithm}
  \caption{Greedy Algorithm to Solve Subset Sum}
  \for{element in set}
\end{algorithm}

One characteristic of a greedy algorithm is that, once we decide to include an element
to the sum, we never undo the decision. In the case of the subset sum problem, this
can easily cause the algorithm to not find the correct list of elements to use for the sum
since not all combinations of integers in a list are necessarily going to add up to
the target sum. Because of this, a margin of error is introduced where the algorithm
may fall short of the target sum. The upside of this algorithm is the time complexity
improvement: the algorithm visits each element of the set only once, giving the
algorithm a tightly bound time complxity of O(n).

\section{Trivial Case where the Greedy Algoithm Fails}
Due to the nature of the greedy algorithm explained in the previous section,
we can identify a trivial case where the algorithm will fail. Take for example
the set of integers S below where the target sum is 150.
$$S = {49, 100, 50}$$
The greedy algorithm will start from the first element and work its way through
the list, adding numbers so long as the running sum does not go beyond the target sum of 150.
In this case, the algorithm will accept the integers 49 and 100, resulting in a sum of 149.
When it encounters 50, it decides not to add this number since it would bring the running
sum over the target sum. The algorithm has then completed processing the set, but it
has failed to find a solution depite the fact one actually exists.

\section{Results}

The results are presented in the figure below. The vertical axis represents the
input size and the horizontal axis the word length of the elements in bits.

%\begin{figure}[h]
%\centering
%\includegraphics[width=12cm]{P3_res.png}
%\caption{Run time for various input size and word length combinations}
%\end{figure}

\end{document}
